\maketitle

So, you're interested in playing in one of my campaigns? Well, there's only one thing I have to say to you.

\section{Welcome!}

I've been playing D\&D and running campaigns for close to 20 years now, and it's always excited to bring someone new into that fold.

This document is meant to give you a brief overview of everything you need to know before we jump into things. It's not a definitive source of truth. Anything I tell you in person, and especially things discussed at session 0, will override what's written here - but this is based on campaigns I've run before, and I think it's a pretty reliable set of thoughts and guidelines.

\begin{DndComment}{Systems aplenty}
 For the purposes of this document, I'm assuming that we're playing Fifth Edition Dungeons and Dragons. Other systems are always on the table, so if you want to try one out - let me know!
\end{DndComment}

\subsection{What to Expect}

The campaigns I run tend to be very light, very silly, and with a strong focus on the story. I'd rather people have a good time than strictly adhere to mechanics, 

\begin{DndQuotation}{Richard, 8 years campaigning together}
TODO GET A QUOTE FROM RICHARD
\end{DndQuotation}

\begin{DndQuotation}{Elizabeth, 2 years campaigning together}
TODO GET A QUOTE FROM ELIZABETH
\end{DndQuotation}

\begin{DndQuotation}{Josh, 6 years campaigning together}
"Run"
\end{DndQuotation}

\subsection{What I Expect}

For the most part social expectations will be discussed as a group, but there are three things that I expect from all players.

\textbf{Reliability} is the key to all succesful RPG campaigns. If you're joining a campaign, make sure we can can rely on your to be at sessions.

Life is complex, and I don't expect 100\% attendance from every single player - but keep us in the loop. If you know you're going to be unavailable, let us know sooner, not later.

\textbf{Engagement} at the table is the second pillar of a good RPG campaign. If you want to play D\&D, then be ready to play D\&D. I'm not going to boot you from the campaign because you checked your phone. What \textit{is} going to cause some tension is getting regularly \textbf{distracted} by your phone.

It's for this reason that I have a \textit{strong} preference for physical character sheets and dice. You're welcome to keep a copy of your character on DnDBeyond or whatever, but when it comes to playtime - please have the sheet in front of you, or at least a PDF copy of the sheet. Anything from dodgy internet to lots of fiddly buttons can slow a session right down, but your sheet has everything on it right there in front of you. 

\textbf{Know how to play your character}. You don't have to read every source book cover to cover, but you should understand the mechanics of the game enough to run your character. It's one thing to have a learning curve - I'm happy to help with that, but you've got to meet me halfway.
