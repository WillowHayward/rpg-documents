\maketitle

So, you're interested in playing in one of my campaigns? Well, there's only one thing I have to say to you.

\section{Welcome!}

I've been playing D\&D for close to 20 years now, and it's always exciting to have someone new at the table.

This document is meant to give you a brief overview of everything you need to know before we jump into things. It's not a definitive source of truth. Anything discussed in person will override what's written here - but this is based on campaigns I've run before, and I think it's a pretty reliable set of thoughts and guidelines.

\begin{DndComment}{Systems aplenty}
 For the purposes of this document, I'm assuming that we're playing Fifth Edition Dungeons and Dragons. Other systems are always on the table, so if you want to try one out - let me know!
\end{DndComment}

\subsection{What to Expect}

The campaigns I run tend to be very light, very silly, and with a focus on the story and the characters. I'll try to challenge you, but I don't run mechanically intense combats, and while "crunchy" rules might make their way in from time to time, they're far from a mainstay.

I like to go into a campaign with a high-level plan for the story - the world, the major players, the big beats. And then once players enter the mix, that plan moves the background. I'll try to tie backstories into the world, swap out major players for ones that the players will connect better with, and even shift around those major beats to better pace things.

I'm comfortable with improv, so making bold decisions isn't just something I encourage, it's something I want. Some of my favourite tabletop moments have been when players surprised and stumped me.

\subsection{What I Expect}

We'll discuss most of this at the table, but there are three things that I expect from all players.

\begin{description}
    \item[Reliability.] If you're joining a campaign, make sure we can can rely on your to be at sessions. Life is complex, and I don't expect 100\% attendance from every single player - but keep us in the loop. If you know you're going to be unavailable, let us know sooner, not later.

    \item[Engagement.] at the table is the second pillar of a good RPG campaign. If you want to play D\&D, then be ready to play D\&D. I'm not going to boot you from the campaign because you checked your phone. What \textit{is} going to cause some tension is getting regularly \textbf{distracted} by your phone.

    \item[Know how to play your character.] You don't have to read every source book cover to cover, but you should understand the mechanics of the game enough to run your character. It's one thing to have a learning curve - I'm happy to help with that, but you've got to meet me halfway.
\end{description}

\subsection {Testimonals}

I reached out to some of my players to see if they had anything to say to someone about to join one of my campaigns.

\begin{DndQuotation}{Josh, 6 years campaigning together}
"Run"
\end{DndQuotation}

\begin{DndQuotation}{Elizabeth, 2 years campaigning together}
"I loved playing with Willow as DM! Her focus is on making sure the game is enjoyable for each of her players and she really leans in to their play style. For me, that meant I got to feel like I was telling an epic story and that fights were all about creative problem solving - aka I got to be a Hestia evangelist who sometimes got used as a halfling flail by my less-height challenged party members!"
\end{DndQuotation}

\begin{DndQuotation}{Richard, 8 years campaigning together}
"don't"
\end{DndQuotation}
