\section {The World}

I'm a worldbuilder at heart. Making custom settings and the lore belonging to them are a big part of why I love RPGs, so the vast majority of games I run will be in homebrew worlds.

Depending on the system and setting of the campaign we're running, there's a handful of different worlds available, but if we're running something fantasy then you can expect to be playing in \textbf{The Cosmoses of Creation} (or CoC for short). 

The CoC is a set of 9 planes (cosmoses, cosmos singular), but the two that you'll spend the most of your time in are called \textbf{Sheneth} and \textbf{Fulsk}.

\subsection {Ask me about it}

I'm working on expanding the library of readily-available worldbuilding content for my campaigns, but if there's something that you're curious about in my worlds - ask me about it. I love talking about this stuff, and if you think something is neat I can try to work it (or something similar) into the campaign. Plus, if you identify a gap in the lore, maybe you can help me fill it.

A lot of the CoC lore is collaborative. If it doesn't come from me, it comes from the players - and the whole project is better for it.

\subsection {A Little Bit Familiar}

The core rulebooks of D\&D are written with a particular setting in mind - The Forgotten Realms. The Cosmoses of Creation are not The Forgotten Realms. Some aspects are similar - Sheneth and Fulsk function similarly to The Material Plane, the other cosmoses can each function as other Great Wheel planes in a pinch, but at the end of the day - they're different settings.

Functionally, this just means that things you see referenced in the rulebooks might not be true in the game. This doesn't actually come up all that often in-session, and world details

However, with all that said - if there's something from The Forgotten Realms that you like or want in your backstory, I'm not going to stop you. You might see something get renamed - Waterdeep becoming Deepwell, Mechanus becoming The Clockwork Gateway - but it's a big cosmoverse, there's room for all sorts of ideas.

\subsection {Languages}

Most D\&D languages in The CoC map directly or indirectly to a language on Earth. Some of these are for worldbuilding reasons, but the majority of them were chosen because a player made a choice that inexorably tied the two languages together.

\begin{DndTable}{XX}
    Fantasy Language  & Earth Language \\
    Common  & English \\
    Gnomish  & Latin \\
    Draconic  & A blend of Slavic \& Eastern European languages  \\
    Celestial  & French \\
    Infernal  & Spanish \\
    Giant  & Gaelic
\end{DndTable}

As you can see, there's a lot of languages left to map - maybe the next one will come from you.

\subsection {Divination}

Something that anyone who studies divination will know is that it works because of an event known as \textbf{The Cacophony}

Before linear time was introduced to most of the cosmoses, they tried to experience existence all at once. Every possible permutation of every possible moment happened simultaneously, trapping existence in a single, infinite moment. That moment is referred to as The Cacophony. They were eventually freed, and the cause-and-effect model of time that we're all familiar with was made standard, but The Cacophony still happened.

The future isn't set - decisions are still being made, and free will exists. But the art of divination is looking \textit{back} in time to The Cacophony, and trying to find the permutations most closely resemble reality today. Done well, this can give you a glimpse into the moments that this moment might lead to.

\subsection {No seriously, ask me about it}

I love talking about this stuff so much. Please ask me questions when you're curious. I will give you a free level up for every question you ask*. Please, I need this.

* I will not
