\section {Characters}

From swashbuckling rogues to heroic warriors to tormented warlocks, there's no limit to the kinds of characters you can make. Except for the limits that I'm about to tell you about - but don't stress, they're very much in place to keep things interesting, to keep the game flowing, and to stop any player from being left behind.

There's three things to consider at this stage of the game - mechanics, story, and play. But above all:

\subsection {Keep Me In The Loop}

While you're making your character, it's a good idea to let me know what you're thinking. Odds are good that you're not going to run into many things I'll veto, but it provides opportunities for me to offer suggestions that can tie your character into the world or the story of the campaign.

\subsection {Paper Sheets}

Call me old school, but I like it when players use paper character sheets. I think it keeps things in the room, and after a couple of sessions you'll know exactly where to look for any bit of information you might need. I'm not going to stop you from building your character on DndBeyond or whatever, but when we're at the table - please have the sheet in front of you, or at least a PDF copy of the sheet. Anything from dodgy internet to lots of fiddly buttons can slow a session right down, but your sheet has everything on it right there in front of you. That said, having an app for spells is fair game.

\section {Mechanics}

\subsection {5e vs 5.5}

In 2024, a new revision of Dungeons and Dragons Fifth Edition was released (colloquially known as D\&D 5.5e). If you want to use 5.5 content for your player, go right ahead.

\subsection {Allowed Content}

When putting together the mechanics of your character, then you can assume that anything officially published by Wizards of the Coast for D\&D 5e or 5.5 is fair game.

In terms of Unearthed Arcana, homebrew, and third party content, I'm usually a little bit more hesistant. So much of it hasn't been playtested, or is balanced with other content from the same module in mind. That's not a hard no, just make sure to ask me first.

That said, feel free to use the races from the third party settings \textit {Humblewood} by Hitpoint Press and \textit{Midgard} by Kobold Press. They're a good source of animalfolk - bears and birds and badgers galore.

\subsection {Stats Generation}

For your first character in a given campaign, you'll be rolling stats (4D6 drop lowest). 

For each stat roll 4d6, take away the lowest roll, and then add the rest together. Do that six times and you'll have six ability scores to be assigned as you please.

You can reroll one of the six for free, and if the sum of all six is less than 70 you can do a full re-roll. Where possible, it's nice to roll your stats in front of the party. I trust my players not to cheat, but it's a good bonding experience.

I like to run it this way because I think having the variation between players makes for interesting moments. The standard array is great for making sure everyone has about the same highs and lows, and point-buy is great for optimising characters - but for the kind of stories we're telling, surrending some consistency helps decouple character from mechanical perfection a bit.

\begin{DndComment}{Meeting Your Character}
In some campaigns, it's nice to come into it without any pre-conceived notions about your character. In aid of this, sometimes when rolling those initial stats, we'll do it \textbf{in order}. First Strength, then Dexterity, then Constitution, etc.

This gives you a unique opportunity to meet your character as you're rolling. If the end result is a character you wouldn't want to play, you can always assign the values elsewhere - but keep an open mind. It's a good way to play characters you might not otherwise think to.
\end{DndComment}

If for whatever reason you end playing a second character in a campaign, you can generate those stats through whatever legal method you prefer.

\section {Story}

\subsection {Backstory}

Backstories can be hard, so the advice I normally give is to write 1-3 paragraphs, and include a reference to a location, a goal, and a character. You don't need all three, and you don't even need to name them, but this gives me as the GM something to hook into.

\begin{DndComment}{Barbarian is not a job}
  It's easy to say that your character is a warlock and leave it at that, but I'd recommend giving some thought as to how that's manifested in their life. What's your character's place in the world? How have they made their living? Did their powers come from the usual routes, or did they find their own way towards the skills of their class? Have some fun with it!
\end{DndComment}

If there's something in your backstory that you don't want me to mess with, let me know. If you establish a backstory spouse you don't want dying - that's something I \textbf{need} to know.

\subsection {Player Secrets}

If there's something about your character that you don't want the other players to know, then that's probably fine. If there's something about your character that you don't want \textit{me} to know, that's gonna cause an issue.

I try to weave character backstories into the world and story, and so if you've decided that the parents of your orphaned hero are the king and queen, and I've decided that they were the evil wizard that's been hunting the party - that's gonna throw off a lot of things for both of us.


\subsection {Beware of Gimmicks}

The Internet is full of all sorts of fun character ideas. A bear using intimidation so that people pretend they're human, three kobolds in a trenchcoat pretending to be one person, one class masquerading as another class.

They're fun to read about, but 9 times out of 10 they're a slog to play. Something that was funny in session 1 might be pure tedium in session 30.

That's not to say you can't have fun - silly ideas have their place, but think about the implications of it. You're gonna spend a lot of time with this character. Don't do anything in creation that you'll come to regret.

\subsection {Alignment is fake}

Alignment is a scam by Big Morality to get us to buy more Ethics.

What I mean by that is - I'm not a huge fan of the traditional 3x3 grid used to describe alignment. With the right justification, most actions could fit into any slot. One person's lawful good is another's lawful evil. So I'm a big believer in removing the abstraction of alignment and cutting through to \textbf{motivation}.

The reasons your character are doing something are more interesting and more important than picking a slot on a 3x3 grid to justify behaviour.

\section {Play}

\subsection {Heroes \& Villains}

%todo - rewrite
There's a strange misconception that I've run into before: Well intentioned heroes aren't interesting. And I don't buy it. Some of the best PCs I've seen have just wanted to do good.

Your character is welcome to be morally grey, maybe even skewing evil, but it all comes back to motivation. The things your character are doing should make sense within the context of who they are, the world they're in, and what they're trying to achieve. I'm not going indulge cruelty for cruelty's sake.

At the end of the campaign, we should have the tale of adventurers who made the world a better place - or died trying.

\subsection {Join The Adventure}

The trope of the closed-off loner is a fantasy classic, but it's important to disguish between the player and the character. Even if all your character wants is to be left alone, as the player you have to make the decision for them to get involved. Obviously I'll try to help with that, but you're the one behind the wheel.

"It's what my character would do" is a noble idea, but everyone at the table is responsible for telling a story. If the call to adventure occurs and your character declines, the adventure is just going to continue without your character in it.

\subsection {Ch-ch-ch-changes}

A corollary to everything else: if you get a few sessions into playing your character and realise there's something that isn't vibing, it's okay to change it. You can even change character entirely. As always, keep me in the loop and we probably won't have any issues.

\subsection {Player vs. Player}

This is less to do with character creation and more to do with the game as a whole, but while we're here - I have a way I like to handle any and all situations where two characters run into a situation where they're in direct opposition. Combat, theft, sometimes even lies - we take a step back from the characters, and look at it as players.

Whoever isn't initiating the conflict gets to decide how it resolves, with eyes on the story. They can decide that whatever is being attempted will succeed, they can decide it will fail, or they can decide to defer to the mechanics and let the dice decide. Simple as that.
